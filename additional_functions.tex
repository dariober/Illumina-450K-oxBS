\subsection{Additional functions} 
\label{sec:additionalFunctions}

\begin{sverbatim} % ____________________________________________________________

aggSamp<- function(X, sampGroup) {
    "Aggregate values in matrix X by sampGroup
    Example:
        aggSamp(X= getBeta(MSet.swan), c('A', 'A', 'B', 'B'))
    " 
    cnames<- sort(unique(sampGroup))
    aggMat<- matrix(data= NA, nrow= nrow(X), ncol= length(cnames))
    colnames(aggMat)<- cnames
    rownames(aggMat)<- rownames(X)
    for(i in 1:length(cnames)){
        xcol<- which(sampGroup == cnames[i])
        aggMat[, i]<- rowMeans(X[, xcol])
    }
    return(aggMat)
}

histPosNeg<- function(x, pval, posBorder= 'firebrick4', 
	freq= TRUE, xlab= 'P-value', ...){
    # Plot two histograms on the same plot for the pvalues of the 
    # positive and negative scores
    # E.g. plot the histogram of pvalues for -ve % hmC and +ve % hmC.
    # x:
    #   Vector of percentages or in general a vetcor of positive and 
    #   negative numbers.
    #   The actual value is not used, only the sign is used to separate pvalues.
    # pval:
    #   pvalues associated to x. The histograms are based on this vector.
    # ...:
    #   Futher args to pass to hist()
    # --------------------------------------------------------------------------
    if (length(x) != length(pval)){
        stop("Different length for x and pval vectors")
    }
    pneg<- pval[which(x <= 0)]
    ppos<- pval[which(x > 0)]
    
    ## y limits of histogrm
    if (freq == TRUE){
        f<- "counts"
    } else {
        f<- "density"
    }
    hneg<- max(hist(pneg, plot= FALSE)[[f]])
    hpos<- max(hist(ppos, plot= FALSE)[[f]])
    hlim<- c(0, max(c(hneg, hpos)))
    
    ## Plot
    hist(pneg, col= 'grey80', border= NA, ylim= hlim, freq= freq, xlab= xlab, ...)
    hist(ppos, col= 'transparent', border= posBorder, add= TRUE, 
    	freq= freq, xlab= xlab, ...)
}

makeTransparent<- function(col, alpha=0.5) {
  # Convert colour spcified as name or in other way to equivalent trasparent.
  # col:
  #    Vector of colours
  # alpha:
  #   Alpha transparency [0, 1]
  # See also (taken from)
  # http://stackoverflow.com/questions/8047668/transparent-equivalent-of-given-color
  #
  if(alpha<0 | alpha>1) stop("alpha must be between 0 and 1")

  alpha = floor(255*alpha)
  newColor = col2rgb(col= col, alpha=FALSE)

  .makeTransparent = function(col, alpha) {
    rgb(red=col[1], green=col[2], blue=col[3], alpha=alpha, maxColorValue=255)
  }

  newColor = apply(newColor, 2, .makeTransparent, alpha=alpha)

  return(newColor)
}


bedtools.intersect<-function(a, b, opt.string=""){
    # Functions to call bedtools tool from R. 
    # Funtions are named bedtools.<tool name>.
    # This is an edit of http://zvfak.blogspot.co.uk/2011/02/calling-bedtools-from-r.html
    # -----------------------------------------------------------------------------

    #create temp files
    a.file<- tempfile(fileext = ".bed")
    b.file<- tempfile(fileext = ".bed")
    out   <- tempfile(fileext = ".bed")
    ori_scipen<- options('scipen')
    options(scipen= 99) # not to use scientific notation when writing out
    
    #write bed formatted dataframes to tempfile
    write.table(a, file= a.file, quote=F, sep="\t", col.names=F, row.names=F)
    write.table(b, file= b.file, quote=F, sep="\t", col.names=F, row.names=F)
    
    # create the command string and call the command using system()
    command<- paste('bedtools intersect', "-a", a.file, "-b", b.file , opt.string, ">",
        out, sep=" ")
    cat(command,"\n")
    try(system(command))
    
    res=read.table(out, header=F, stringsAsFactors= FALSE)
    unlink(a.file);unlink(b.file);unlink(out)
    options(scipen= ori_scipen)
    return(res)
}
\end{sverbatim} % --------------------------------------------------------------